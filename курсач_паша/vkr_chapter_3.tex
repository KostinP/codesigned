\newpage
\setcounter{subsection}{0}
\setcounter{equation}{0}
%\setcounter{section}{0}

\begin{center}
\section{Глава 3. Реализация системы}
\end{center}

\subsection{Описание инфраструктуры}
Docker Compose / Kubernetes, CI/CD, автоматизация деплоя

\begin{figure}[ht!] %[h!]
\begin{center}
\includegraphics[scale=0.5]{fig_template}
\end{center}
\label{fig1}
\caption{Пример анализа данных}
\end{figure}

Таким образом, на основе предложенных систем, включая информационно-аналитические 
(если рассматривать управление данными), рассматриваются системы анализа 
данных (анализа, управления, автоматизации, процессов, систем и т.д.), 
включая, например, системы управления, а также информационные системы.

\subsection{Backend}

Модули (user, course, lab, environment, notification).

API (REST/gRPC).
\subsection{LMS-интеграция}
(привязка к курсам, отслеживание прогресса).

\subsection{Telegram Bot}

Авторизация через Telegram OAuth.

Уведомления (запуск окружения, завершение, ошибки).

Управление окружением (запуск/остановка через чат).

\subsection{Frontend: интерфейс LMS для запуска лабораторных окружений}

\subsection{Примеры сценариев работы (регистрация, запуск лабораторной, проверка результата)}
