\newpage
\setcounter{subsection}{0}
\setcounter{equation}{0}
\setcounter{section}{0}

\begin{center}
\section{Глава 1. Анализ предметной области}
\end{center}

\subsection{Обзор существующих LMS с виртуализацией}

\subsection{Введение}
Системы управления обучением (Learning Management Systems, LMS) являются базовой инфраструктурой 
для организации образовательного процесса в цифровой среде. 
Их основное назначение — хранение и предоставление учебных материалов, управление курсами и пользователями, проведение тестов и оценивание. 
Однако в условиях стремительного развития IT-образования особое значение приобретают LMS, 
интегрированные с виртуальными лабораториями и средами. Подобные решения позволяют студентам выполнять практические задания 
в максимально приближенных к реальным условиях: запускать виртуальные машины, работать с контейнерами, 
моделировать сетевые инфраструктуры и тестировать программные продукты в изолированных средах без риска для основной системы.

\subsection{Примеры решений}
\begin{enumerate}
    \item **Moodle** — одна из наиболее распространённых open-source LMS. 
Благодаря модульной архитектуре поддерживает плагины для интеграции с системами виртуализации: OpenStack, VirtualBox, Docker. 
% Например: https://moodle.org/plugins/mod_virtualpc. Недостаток плагинов в том, что нужно задействовать администратора для установки. 
% В частности этот плагин помимо настройки администратором требует регистрации в стороннем сервисе https://udsenterprise.com/.
% Существуют решения, позволяющие преподавателю автоматически развертывать лабораторные окружения для студентов, 
% однако их внедрение требует значительных технических ресурсов и администрирования. Например: https://moodle.org/plugins/assignsubmission_mojec?lang=en
% Недостаток этого решения в том, что для проверки кода нужно прикреплять файлы в формате zip.
%     \item **Stepik** — российская образовательная платформа, ориентированная на онлайн-курсы. 
    Поддерживает запуск кода в изолированных песочницах, однако возможности полноценной виртуализации ограничены.
    \item **Canvas LMS** — современная платформа с акцентом на удобный интерфейс и интеграцию с внешними сервисами. 
    Виртуализация как таковая не встроена, но через API возможно подключение облачных лабораторий.
    \item **Open edX** — активно используется университетами и корпоративными центрами. 
    Имеет плагины и расширения для запуска практических заданий в песочницах и удалённых средах, 
    что делает платформу более гибкой для технического образования.
    \item **Cisco NetAcad / NetLab+** — специализированные решения для обучения сетевым технологиям. 
    Поддерживают моделирование сетевых топологий, настройку маршрутизаторов и коммутаторов в виртуальных лабораториях.
    \item **AWS Academy и Google Cloud Training** — примеры коммерческих решений, 
    где студенты получают доступ к реальным облачным виртуальным машинам и сервисам. 
    Такие подходы позволяют работать в боевых условиях, но требуют оплаты ресурсов и стабильного интернет-соединения.
\end{enumerate}

**Сравнительный анализ.**

* По уровню интеграции виртуализации можно выделить специализированные решения (NetLab+, AWS Academy), где лабораторная работа является центральным элементом, и универсальные LMS (Moodle, Canvas, Open edX), в которых виртуализация подключается как дополнение.
* Поддержка контейнерных технологий наиболее развита в современных open-source платформах, где активно применяются Docker и Kubernetes.
* С точки зрения стоимости Moodle и Open edX выгодно отличаются, так как являются open-source, но требуют самостоятельной настройки и сопровождения. Коммерческие решения (AWS Academy, Cisco NetAcad) обеспечивают высокое качество и готовую инфраструктуру, однако связаны с дополнительными расходами.
* Гибкость и расширяемость наиболее высока у систем с открытым кодом, что делает их привлекательными для университетов и корпоративных образовательных центров.
* Геймификация в существующих системах, как правило, реализована минимально: чаще всего ограничивается бейджами и рейтингами, без глубокой интеграции с практическими лабораторными заданиями.

**Вывод.**
Современные LMS демонстрируют разнообразие подходов к интеграции виртуализации. Одни делают акцент на специализированных лабораториях для определённых областей знаний, другие предоставляют лишь возможность подключения внешних сервисов. Общим ограничением является разрыв между образовательным процессом и виртуальной средой: студент вынужден работать с несколькими системами параллельно, что снижает удобство и эффективность обучения. На сегодняшний день отсутствует комплексное решение, которое бы одновременно сочетало в себе LMS, виртуализацию и развитую систему геймификации, что формирует актуальную нишу для разработки новых платформ.


(Moodle, OpenEdX, Stepik, Canvas)

\subsection{Подходы к организации лабораторных работ в LMS}

\subsection{Обзор технологий виртуализации и контейнеризации}
(Docker, Docker Compose, Kubernetes).

\subsection{Существующие решения для динамического развертывания окружений}
(Katacoda, Play With Docker, Gitpod).

\subsection{Анализ Telegram API и его применения для образовательных платформ}

\subsection{Постановка проблемы и формулировка требований к системе}
