%%% Поля и размеры страницы %%%
\usepackage{lscape} % Для альбомной ориентации страниц
\usepackage{geometry} % Для конфигурации размеров полей

%%% Математические пакеты %%%
\usepackage{braket,amssymb,amsmath,array,commath,mathtext,wrapfig,tikz}
\usepackage[arrows=pgf-filled,version=3]{mhchem}
\usepackage{bm} % bold math

%%% Форматирование текста %%%
\usepackage{indentfirst} % Отступ абзаца
\usepackage{setspace} % Пакет для регулировки межстрочного интервала
\onehalfspacing % Полуторный интервал для всего текста
%см. далее

%%% Работа с графикой %%%
\usepackage{epsfig,mathtext} % Для работы с графикой и текстом
\usepackage[T2A]{fontenc} % Кодировка шрифта
% \IfFileExists{pscyr.sty}{\usepackage{pscyr}}{} % Подключение шрифта Times New Roman, если установлен пакет PSCyr
%
% https://github.com/AndreyAkinshin/Russian-Phd-LaTeX-Dissertation-Template/tree/master/PSCyr
%
%\usepackage{pscyr} % Подключение шрифта Times New Roman
%\usepackage{TM}
\usepackage[utf8]{inputenc}  % Кодировка входного текста UTF-8
\usepackage[english,russian]{babel} % Языки: английский, русский

%%% Цвет %%%
\usepackage{xcolor}  %\textcolor{red}{цветной текст}

%%% Пагинация страниц %%%
\usepackage{afterpage} %% Подключение страниц для работы с разворотами
\usepackage{totcount}  % Подсчёт общего количества страниц и т.д.
\usepackage[singlelinecheck=off]{caption}	% Конфигурация подписей
\usepackage{soul}	% Поддержка зачёркивания и выделения текста \sout{}.
\usepackage{ulem}

%%% Библиография %%%
\usepackage{cite} % Поддержка ссылок на литературу
\usepackage[nottoc,notlot,notlof]{tocbibind}

%%% Графика %%%
\usepackage{graphicx} %
\usepackage{subfigure} % Поддержка нескольких изображений для одной подписи!!

%%% Таблицы %%%
\usepackage{dcolumn} % Выравнивание десятичных чисел по разделителю.
\usepackage{changebar}
\usepackage{longtable} % Длинные таблицы.
\usepackage{hhline} % Горизонтальные линии для таблиц (\hhline{|=|~~|-|}).
\usepackage{multirow,makecell,array} % Поддержка многострочных таблиц.

\usepackage{float}