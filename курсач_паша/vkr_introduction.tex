\newpage
\setcounter{subsection}{0}
\setcounter{equation}{0}
\setcounter{section}{0}

\begin{center}
\section{Введение}
\end{center}

В условиях цифровизации образования особое значение приобретает применение современных информационных технологий, 
направленных на повышение качества и эффективности учебного процесса. 
Одним из ключевых направлений является развитие систем управления обучением (Learning Management Systems, LMS), 
которые обеспечивают централизованное хранение учебных материалов, контроль знаний обучающихся и автоматизацию организационных процессов.

Однако существующие LMS (в частности Moodle) преимущественно ориентированы на управление контентом и из коробки не обладают достаточными средствами 
для организации выполнения лабораторных работ, требующих специализированного программного или аппаратного окружения. 
В то же время выполнение лабораторных заданий является неотъемлемым компонентом подготовки специалистов в области информационных технологий, 
так как позволяет формировать практические навыки работы с реальными системами.

Традиционная организация лабораторных занятий в компьютерных классах сопряжена с рядом проблем: 
необходимостью предварительной настройки оборудования и программного обеспечения, 
значительными трудозатратами со стороны преподавателей и системных администраторов, 
ограниченностью ресурсов и высокой стоимостью их поддержки. 
В этой связи актуальным направлением исследований становится разработка механизмов динамического развертывания виртуальных учебных сред, 
которые могут предоставляться студентам по запросу и использоваться для выполнения практических заданий.

Современные технологии виртуализации и контейнеризации (в частности, Docker и Kubernetes) позволяют автоматизировать процессы создания, 
конфигурирования и управления изолированными окружениями. Это открывает возможности для интеграции подобных решений в LMS, 
что позволит повысить гибкость образовательного процесса и обеспечить масштабируемость инфраструктуры.

Дополнительное значение приобретает использование средств мгновенного обмена сообщениями, таких как Telegram, 
которые обладают развитым API и позволяют организовать удобное взаимодействие с пользователями. 
Интеграция LMS с Telegram-ботом обеспечивает оперативное информирование обучающихся, 
доступ к лабораторным заданиям и возможность взаимодействия с виртуальными средами в привычном для студентов интерфейсе.

Таким образом, актуальность исследования обусловлена необходимостью разработки системы, 
обеспечивающей динамическое развертывание виртуальных лабораторных окружений в составе LMS с использованием Telegram API.

Цель исследования – разработка системы динамического развертывания виртуальных сред для выполнения лабораторных работ в составе LMS 
с использованием Telegram API.

Для достижения поставленной цели необходимо решить следующие задачи:
\begin{enumerate}
    \item Провести анализ существующих LMS и решений в области автоматизированного развертывания виртуальных сред.
    \item Определить требования к системе и разработать её архитектурную модель.
    \item Реализовать прототип системы, включающий модуль LMS, сервис управления виртуальными окружениями и компонент взаимодействия с Telegram API.
    \item Обеспечить интеграцию с LMS для запуска лабораторных окружений и фиксации результатов выполнения заданий.
    \item Провести тестирование разработанной системы и оценить её эффективность.
\end{enumerate}

Объект исследования – процессы организации и выполнения лабораторных работ в системе дистанционного обучения.
Предмет исследования – методы автоматизации развертывания виртуальных учебных окружений и их интеграция в состав LMS.

Практическая значимость работы заключается в возможности применения разработанной системы в образовательных организациях 
для снижения трудозатрат на подготовку лабораторных занятий, повышения доступности практико-ориентированного обучения 
и улучшения качества подготовки специалистов в области информационных технологий.

